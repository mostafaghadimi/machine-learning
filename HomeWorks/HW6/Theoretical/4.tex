\سؤال{یادگیری جمعی}

\begin{itemize}
	\item الف)
	نامساوی \lr{Jensen}:
	
	$$
	f(\Sigma_{i = 1}^M\lambda_i x_i) \leq \Sigma_{i = 1}^M\lambda_if(x_i)
	$$
	همان‌طور که در صورت سوال گفته شده است برای $E_{avg}$ داریم:
	$$
	E_{avg} = \frac{1}{M} \Sigma_{i = 1}^{M}E_x[(h_m(x) - h(x)^2]
	$$
	حال اگر $\frac{1}{M}$ را به درون سیگما ببریم، داریم:
	$$
	E_{avg} = E_x[\Sigma_{m=1}^M  \frac{1}{M}(h_m(x) - h(x))^2]
	$$
	
	حال با توجه به نامساوی \lr{Jensen} و محدب بودن تابع داریم:
	$$
	(\Sigma_{m=1}^M\frac{1}{M}(h_m(x) - h(x))^2 \leq \Sigma_{m=1}^M\frac{1}{M}(h_m(x) - h(x))^2
	$$
	$$
	\implies E_{com} \leq E_{avg}
	$$
	\item ب)
	
	$$
	E_{avg} = \frac{1}{M} \Sigma_{m=1}^M E_x[(h_m(x) - h(x))^2]
	$$
	
	$$
	E_{com} = E_x[(\frac{1}{M} \Sigma_{m=1}^M h_m(x) - h(x))^2] = E_x[(\frac{1}{M} \Sigma_{m=1}^M h_m(x) - h(x)) (\frac{1}{M} \Sigma_{l=1}^M h_l(x) - h(x))]
	$$
	حال اگر یکی از عامل‌های $\frac{1}{M}$ را به دلیل ثابت بودن از داخل امید ریاضی بیرون بیاوریم، به دلیل فرضیات یعنی
	 $$
	 \forall m \neq l \: E[(h_m(x) - h(x))(h_l(x)-h(x))] = 0
	 $$
	 داریم:
	 $$
	 E_{com} = \frac{1}{M}(\frac{1}{M} \Sigma_{m=1}^M E_x[(h_m(x) - h(x))^2]) = \frac{1}{M}E_{avg}
	 $$
\end{itemize}
