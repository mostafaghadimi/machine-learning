\سؤال{مقدار ویژه و رنک}
\begin{itemize}
	\item ثابت کنید اگر $P$ یک ماتریس با رنک کامل باشد، ماتریس‌های $M$ و $P^{-1}MP$ مجموعه مقدار ویژه‌ی یکسان دارند.
	
	فرض کنید $\lambda$ محموعه مقدار ویژه‌ی $P^{-1}MP$ است. باید نشان دهیم که $\lambda$ مقدار ویژه‌ی $M$ نیز می‌باشد. حال اگر $x$ بردار ویژه باشد داریم:
	$$P^{-1}MPx = \lambda x \rightarrow M(Px) = \lambda Px$$
	
	بنابراین $Px$ یک بردار ویژه با مقدار ویژه‌ی $\lambda$ است. 
	به طور مشابه طرف دیگر آن اثبات می‌شود. اگر بردار ویژه‌ی $x$ متعلق به $M$ باشد، می‌توان نشان داد که $P^{-1}x$ یک بردار ویژه از $P^{-1}MP$ است:
	
	$$P^{-1}MP(P^{-1}x) = P^{-1}Mx = \lambda P^{-1}x$$
	ثابت شد که مقدار ویژه‌ی یکسان دارند. 
	
\textbf{نکته}: 
از خاصیت رنک کامل بودن برای معکوس‌پذیر بودن $P$ استفاده شده است.
\item ثابت کنید که مجموع ابعاد فضای ویژه ماتریس $M_{n\times n}$ نمی‌تواند بیش‌تر از $n$ شود. ($M$ در بیش‌ترین حالت $n$ تا مقدار ویژه‌ی متفاوت دارد.)


\end{itemize}