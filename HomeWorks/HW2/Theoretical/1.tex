\سؤال{مقدار ویژه}

\begin{itemize}
	\item جمع مقادیر ویژه برابر با \lr{trace}  است.
	
	اگر معادله مشخصه ماتریس را بنویسیم (
	$p(\lambda) = det(A - \lambda I)$)، 
	داریم:
	$$ p(\lambda) = \det(A - \lambda I) = (-1)^n \big(\lambda^n - (\text{tr} A) \, \lambda ^{n-1} + \dots + (-1)^n \det A\big)\,.$$
	 
	 از طرفی می‌دانیم:
	 
	 $$p(\lambda) = (-1)^n(\lambda - \lambda_1) \dots (\lambda - \lambda_n)$$
	 
	 $\lambda_j$ نشان‌دهنده‌ی مقدار ویژه‌ی ماتریس $A$ است؛ بنابراین با مقایسه‌ی ضرایب، این رابطه اثبات می‌شود.
	 
	 $$tr(A) = \lambda_1 + \dots + \lambda_n$$
	 
	 \item ضرب مقادیر ویژه برابر با دترمینان ماتریس است.
	 
	 بار دیگر اگر معادله‌ی مشخصه را بنویسیم و فرض کنیم که $\lambda$ ریشه‌های این معادله هستند، داریم:
	 
	 $$
	 \begin{array}{rcl} \det (A-\lambda I)=p(\lambda)&=&(-1)^n (\lambda - \lambda_1 )(\lambda - \lambda_2)\cdots (\lambda - \lambda_n) \\ &=&(-1) (\lambda - \lambda_1 )(-1)(\lambda - \lambda_2)\cdots (-1)(\lambda - \lambda_n) \\ &=&(\lambda_1 - \lambda )(\lambda_2 - \lambda)\cdots (\lambda_n - \lambda)
	 \end{array}
	 $$
	 
	 در معادله‌ی اول، ضریب $(-1)^n$ از فاکتورگیری به دست آمده است. این ضریب را می‌توان به‌جای فاکتورگیری روی قطر گسترش داد و به معادله‌ی دوم رسید. 
	 
	 حال اگر $\lambda$ را برابر با صفر قرار دهیم (چون متغیر است)،‌ در سمت چپ معادله‌ی مشخصه $det(A)$ و در سمت راست $\lambda_1 \dots \lambda_n$ به دست می‌آید، به بیان دیگر:
	 
	 $$det(A) = \lambda_1 \dots \lambda_n$$
	 
	 \item مقدار ویژه‌های $AB$ و $BA$ با یک‌دیگر برابرند.

	 اگر $\lambda$ مقادیر ویژه‌ی $AB$ باشند و آن را در بردار ناصفر $x$ ضرب کنیم، داریم:
	 
	 $$ABx = \lambda x$$
	 
	 حال اگر $y = Bx$ باشد، آن‌گاه $y$ هم ناصفر خواهد بود (در غیر این صورت در معادله‌ی بالا یا $x$ و یا $\lambda$ برابر با صفر می‌شوند که خلاف فرضیات ما است.). با جای‌گذاری روابط داریم:
	 
	 $$BAy=BABx=B(ABx)=B( \lambda x)=\lambda Bx = \lambda y$$
	 
	 همان‌طور که مشاهده می‌شود، ثابت شد که $\lambda$ مقدار ویژه‌ی $BA$ نیز هست.
\end{itemize}


