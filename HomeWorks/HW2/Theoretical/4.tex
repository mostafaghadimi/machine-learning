\سؤال{معین متقارن}

ثابت کنید هر ماتریس مثبت معین متقارن $A$ یک شکل یکتا به فرم $A = LL^T$ دارد که $L$ پایین مثلثی با درایه‌های قطری مثبت است. (چولسکی)

برای اثبات آن از استقرا روی ابعاد آن استفاده می‌کنیم:
\begin{itemize}
	\item اگر $n = 1$: 
	ماتریس
	 $A_{1 \times 1}$ مثبت معین است؛ اگر و تنها اگر به فرم 
	 $A = (a)$ با $a > 0$  باشد. 
	 اگر $l = \sqrt{a}$ و $L = (l)$ باشد، جلوتر ثابت می‌کنیم که $L$ فاکتور چولسکی $A$ می‌باشد.
	 \item
	 اگر 
	 $n \rightarrow n + 1$:
	 فرض کنید برای همه‌ی ماتریس‌های متقارن و مثبت معین نشان داده‌ایم که فاکتور چولسکی دارند؛ بنابراین ماتریس $A$ را می‌توان به فرم زیر نوشت:
	 $$A = 
	 \begin{pmatrix}
	 a_{11} & A_{12}\\
	 A_{21} & A_{22}\\
	 \end{pmatrix}$$
	  که
	 $a_{11} \in \mathbb{R}$، 
	 $A_{12} \in \mathbb{R}^{1\times n}$،
	 $A_{21} = A_{12}^T$ و 
	 $A_{22} \in \mathbb{R}^{n \times n}$ است. حال تعریف $S$ را به صورت زیر تعریف می‌کنیم:
	 
	 $$S = A_{22} - \frac{1}{a_{11}}A_{21}A_{12}$$ 
	 $S$، مکمل \lr{Schur} ماتریس $A$ با توجه به $a_{11}$ است. 
	 بدیهی است که ماتریس $S$ متقارن است. بنابراین تنها کافی است نشان دهیم که مثبت معین است. فرض کنید $x \neq 0$ و 
	 	 $y \in \mathbb{R}^{n \times n}$ 
	 بوده و به صورت زیر باشد:
	 $$ y = 
	 \begin{pmatrix}
	 -\frac{1}{a_{11}}A_{12}x\\
	 x
	 \end{pmatrix}$$
	 آن‌گاه $y \neq 0$ است و بنابراین با مثبت معین بودن $A$ ثابت می‌کند که $y^TAy > 0$. علاوه‌بر آن داریم:
	 $$y^TAy = 
	 \begin{pmatrix}
	 -\frac{1}{a_{11}}A_{12}x & x^T \\
	 \end{pmatrix}
	 \begin{pmatrix}
	 a_{11} & A_{12} \\
	 A_{21} & A_{22}\\
	 \end{pmatrix}
	 \begin{pmatrix}
	 -\frac{1}{a_{11}}A_{12}x \\
	 x\\
	 \end{pmatrix}
	 $$
	 $$ =
	 \begin{pmatrix}
	 -\frac{1}{a_{11}}A_{12}x & x^T \\
	 \end{pmatrix}
	 \begin{pmatrix}
	 -A_{12}x + A_{12}x\\
	 -\frac{1}{a_{11}}A_{21}A_{12}x + A_{22}x
	 \end{pmatrix}
	 $$
	 $$ = 
	 \begin{pmatrix}
	 -\frac{1}{a_{11}}A_{12}x & x^T \\
	 \end{pmatrix}
	  \begin{pmatrix}
	 0\\
	 -\frac{1}{a_{11}}A_{21}A_{12}x + A_{22}x
	 \end{pmatrix}
	 $$
	 $$ = 
	 - \frac{1}{a_{11}}x^TA_{21}A_{12}x + x^TA_{22}x
	 $$
	 $$= x^TSx$$
	 بنابراین $x^TSx > 0$ هرگاه $x \neq 0$  ثابت می‌شود که $S$ مثبت معین است. با توجه به متقارن و مثبت معین بودن ماتریس $S$ و با استفاده از فرض استفرا می‌توان نتیجه گرفت که تجزیه چولسکی دارد؛ یعنی می‌توان آن را به صورت $S = L_SL_S^T$ نوشت و $L$ ماتریس پایین مثلثی است.
	 $L$ را به شکل زیر تعریف می‌کنیم:
	 $$L = 
	 \begin{pmatrix}
	 \sqrt{a_{11}} & 0 \\
	 \frac{1}{\sqrt{a_{11}}}A_{21} & L_S
	 \end{pmatrix}
	 $$
	 مثبت معین بودن $A$ به ما می‌فهماند که $a_{11} > 0$؛ بنابراین با استفاده از $A_{21}^T = A_{12}$ به راحتی می‌توان دید که $L$ پایین مثلثی است. 
	 $$LL^T = 
	 \begin{pmatrix}
	 \sqrt{a_{11}} & 0 \\
	 \frac{1}{\sqrt{a_{11}}}A_{21} & L_S
	 \end{pmatrix}
	  \begin{pmatrix}
	 \sqrt{a_{11}} & 	 \frac{1}{\sqrt{a_{11}}}A_{12} \\
	 0 & L_S^T
	 \end{pmatrix}
	 $$
	 $$= 
	 \begin{pmatrix}
	 a_{11} & A_{12} \\
	 A_{21} & \frac{1}{a_{11}}A_{21}A_{12} + L_SL_S^T
	 \end{pmatrix}
	 $$
	 $$ = 
	  \begin{pmatrix}
	 a_{11} & A_{12} \\
	 A_{21} & \frac{1}{a_{11}}A_{21}A_{12} + S
	 \end{pmatrix}
	 $$
	 $$ = A$$
\end{itemize}