\سؤال{مشتق‌گیری}

هر کدام از روابط زیر را ثابت کنید.
\begin{itemize}
	\item 
	$\frac{\partial a^tx}{\partial x} = a$
	
	$$
	\frac{\partial a^tx}{\partial x} = 
	\frac{\partial \begin{pmatrix}
		a_1 & \dots & a_n \\
		\end{pmatrix}
		\begin{pmatrix}
		x_1 \\
		\vdots \\
		x_n \\
		\end{pmatrix}}{\partial x}
	= \frac{\partial (a_1x_1 + \dots + a_nx_n)}{\partial x}
	$$
	اگر $f = (a_1x_1 + \dots + a_nx_n)$ داریم:
	$$
	\frac{\partial a^tx}{\partial x} = 	\frac{\partial f}{\partial x} = 
	\begin{pmatrix}
		\frac{\partial f}{\partial x_1} \\
		 \dots \\ \frac{\partial f}{\partial x_n}\\
	\end{pmatrix} =
	\begin{pmatrix}
	a_1 \\
	\dots \\
	 a_n \\
	\end{pmatrix}
	= a
	$$
	
	\item 
	$\frac{\partial x^tAx}{\partial x} = 2xA$
	
	$$
	y = f(x) = x^TAx = \Sigma_{i = 1}^{n}\Sigma_{j = 1}^{n}a_{ij}x_{i}x_{j}
	$$
	$$
	= \Sigma_{i = 1}^{n}a_{i1}x_{i}x_{1} + \Sigma_{j = 1}^{n}a_{1j}x_{1}x_{j} + \Sigma_{i = 2}^{n}\Sigma_{j = 2}^{n}
	$$
	$$
	\frac{\partial f}{\partial x_{1}} = 
	\Sigma_{i = 1}^{n}a_{i1}x_{i} + \Sigma_{j = 1}^{n}a_{1j}x_j
	$$
	از آن‌جایی که $a_{1i} = a_{i1}$ داریم:
	$$
	= \Sigma_{i = 1}^{n}a_{1i}x_{i} + \Sigma_{i = 1}^{n} a_{1i}x_{i}
	$$
	$$
	= 2 \Sigma_{i = 1}^{n}a_{1i}x_i
	$$
	$$
	\rightarrow \frac{\partial f}{\partial x} = \begin{pmatrix}
	2\Sigma_{i = 1}^{n}a_{1i}x_{i} \\
	\vdots \\
		2\Sigma_{i = 1}^{n}a_{ni}x_{i}
	\end{pmatrix}
	= 2 \begin{pmatrix}
	a_{11} & \dots & a_{1n}\\
	\vdots & \ddot & \vdots \\
	a_{n1} & \dots & a_{nn}
	\end{pmatrix}
	\begin{pmatrix}
	x_1 \\
	\vdots \\
	x_n \\
	\end{pmatrix}
	= 2Ax = 2x^TA
	$$ 
	\item 
	$\frac{\partial^2 x^TAx}{\partial x \partial x^T} = A + A^T$
	
	با توجه به قاعده‌ی ضرب در بردارها داریم:
	$$\frac{\partial u^Tv}{\partial x} = u^T \frac{\partial v}{\partial x} + v^T \frac{\partial u}{\partial x}$$
	
	برای راحتی کار در قسمت بعد رابطه‌ی $\frac{\partial Ax}{\partial x} = A$ را در این قسمت ثابت می‌کنیم:
	$$
		s = Ax \rightarrow s_i = \Sigma_{j}a_{ij}xj \rightarrow \frac{\partial s_i}{\partial x_j} = a_{ij} \rightarrow \frac{\partial s}{\partial x} = A
	$$
حال اگر قاعده‌ی بالا را استفاده کنیم، داریم:
$$\frac{\partial x^TAx}{\partial x}
= x^T \frac{\partial Ax}{\partial x} + (Ax)^T \frac{\partial x}{\partial x} = x^TA + (Ax)^T = x^T(A + A^T)
$$ 

$$ \rightarrow \frac{\partial^2 x^TAx}{\partial x \partial x^T} = \frac{\partial x^T(A + A^T)}{\partial x^T} = A + A^T$$

\item $\frac{\partial tr(X^TAX)}{\partial X} = X^T(A + A^T)$ 

$$(X^TAX)_{ij} = \Sigma_{k}\Sigma_{l}x_{ki}a_{kl}x_{lj}$$
$$tr(X^TAX) = \Sigma_{r}\Sigma_{k}\Sigma_{l} x_{kr}a_{kl}x_{lr}$$
$$\frac{\partial tr(X^TAX)}{\partial x_{ij}} = \Sigma_{l}a_{il}x_{lj} + \Sigma_{k} x_{kj}a_{ki} = \Sigma_{k}x_{kj}a_{ki} + \Sigma_{l}x_{lj}a_{il}$$
$$\rightarrow \frac{\partial tr(X^TAX)}{\partial X} = X^TA + X^TA^T = X^T(A + A^T)$$
\end{itemize}