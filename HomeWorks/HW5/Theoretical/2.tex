\سؤال{}

همان‌طور که در اسلاید‌ها
\footnote{اسلاید ۷، صفحه ۱۷}
دیدیم، داریم:
\newline
\newline
\textbf{تعریف محدب بودن:}
به ازای هر دو نقطه دل‌خواه مانند \lr{A} و \lr{B}، که متعلق به ناحیه‌ی $R_C$ هستند، خط متصل‌کننده‌ی آن دو نقطه نیز باید کاملا در آن ناحیه قرار داشته باشد.



$$
\forall \alpha \in [0, \: 1] \rightarrow
\hat{x} = \alpha x_A + (1 - \alpha) x_B
$$

$$
F_C(x) = F_C(\alpha A + (1 - \alpha) B) \xRightarrow{linearity \: of \: F} \alpha F_C(A) + (1 - \alpha) F_C(B)
$$

$$
\rightarrow \forall d \in {1, \: ..., \: k}: F_C(A) \geq F_d(A), \: F_C(B) \geq F_d(B) \implies \alpha F_C{A} \geq \alpha F_d(A), \: (1 - \alpha) F_C(B) \geq (1 - \alpha) F_d(B)
$$
با توجه به نتیجه‌ی بالا، از جمع کردن دو طرف نامساوی داریم:
$$
\forall d \in {1, \: ..., \: k}: \alpha F_C(A) + (1 - \alpha) F_C(B) \geq \alpha F_d(A) + (1 - \alpha)F_d(B)
$$

پس ثایت می‌شود که به ازای هر نقطه‌ی روی خط متصل‌کننده‌ی $A$ و $B$ داریم:

$$
F_C(\theta) \geq F_d(\theta)
$$

پس ثابت شد که $\theta$ نیز در ناحیه‌ی $R_C$ است. بنابراین محدب بودن به این شکل ثابت می‌شود.