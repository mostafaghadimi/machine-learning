\سؤال{}

\begin{itemize}
	\item اگر \lr{validation}ها از توزیع‌های متفاوتی باشند، در نمودار پس از چندین مرحله \lr{validation 1} مقدار خطایش از حالت نزولی به حالت صعودی تغییر پیدا می‌کند و آن را می‌توان به‌خاطر \lr{overfit}  شدن بعد از بهبود پارامترها توجیه کرد و نمودار \lr{validation 2} هم به توزیع داده‌ی \lr{train} نزدیک است. 

	\item اگر توزیع‌ \lr{validation}هایکسان باشند، تنها روشی که می‌توان این نمودار را توجیه کرد این است که داده‌ها بسیار کم باشند، زیرا در حالت کلی، هنگامی که داده‌ها زیاد باشد، به‌طور میانگین نمودار  \lr{validation}ها باید یکسان باشد.
\end{itemize}