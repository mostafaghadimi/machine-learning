\سؤال{}

چون مقدار $f(x)$ را به طور مستقل برای هر $x$ می‌توان انتخاب کرد، مینیمم $L_q$ را می‌توان با کمینه کردن انتگرال زیر به دست آورد:

$$
 \int |f(x) - y|^q p(y|x) dy
$$

حال اگر نسبت به $f(x)$ مشتق بگیریم و آن را برابر با صفر قرار دهیم، داریم:

$$
 \int q |f(x) - y|^{q - 1} sign(f(x) - y)p(y|x)dy
= q\int_{-\infty}^{f(x)} |f(x) - y|^{q - 1} p(y|x)dy - q \int_{f(x)}^{\infty} |f(x) - y|^{q - 1}p(y|x)dy = 0
$$

$$
\rightarrow \int_{-\infty}^{f(x)} |f(x) - y|^{q - 1} p(y|x)dy = \int_{f(x)}^{\infty} |f(x) - y|^{q - 1}p(y|x)dy
$$

اگر $q = 1$:
$$
\int_{-\infty}^{f(x)} p(y|x)dy = \int_{f(x)}^{\infty} p(y|x)dy
$$
بنابراین همان‌طور که از تعریف می‌دانیم و در سوال ۱ به آن پرداختیم، در میانه (در حالت پیوسته) مساحت قسمت سمت چپ و سمت راست با یک‌دیگر برابر است، یعنی $f(x)$ باید میانه باشد. 

اگر $q \rightarrow 0$:

مقدار $|f(x) - y|^q$ بسیار نزدیک ۱ می‌شود (به جز همسایه‌های کوچکی نزدیک $f(x) = y$ که مقدار آن صفر می‌شود.).
بنابراین مقدار 
$
\int |f(x) - y|^q p(y|x) dy
$
نزدیک ۱ می‌شود؛ زیرا $p(y)$ نرمال شده  است اما مقدار آن کمی در نزدیکی $y = f(x) $ کاهش پیدا می‌کند. بیش‌ترین کاهش مربوط به همین نقطه است که بیش‌ترین $p(y)$ را دارد. 